\documentclass[fontsize=11pt]{article}
\usepackage{amsmath}
\usepackage[utf8]{inputenc}
\usepackage[margin=0.75in]{geometry}

\title{CSC110 Project Proposal: Visualizing the Effects of Rising Sea Levels}
\author{Yousuf Hassan, Aaditya Mandal, Faraz Hossein, Dinkar Verma}
\date{Monday, December 14, 2020}

\begin{document}
\maketitle

\section*{Problem Description and Research Question}

\textbf{Research question: What will the magnitude of the global mean sea-level rise look like from 1993 to 2100?} \\

Global warming is the main contributor to rising sea-levels. More specifically, we can narrow it down to three factors which are the main contributors to rising global mean sea levels. Ocean heat contents, ice sheets melting, and glaciers melting (Church et al. 1143).\\

Ocean heat content is the amount of heat stored in the ocean, measured in Joules. In fact, about 90\% of the extra heat on Earth is stored in the ocean (Dahlman and Lindsey). Rises in ocean temperature are proportional to rises in global sea levels. As the ocean becomes warmer, the process of thermal expansion takes place. The water expands and increases its volume, thus, taking up more space. The amount of thermal expansion that takes place varies on the region; for this project, we are only concerned with the global mean sea level rise, so we take the global ocean heat capacities.\\

Another contributing factor to sea level rises is melting ice sheets, mainly found in Greenland and the Antarctic. Ice sheets are masses of glacial land ice that exceeds over 50,000 square kilometres. The Antarctic and Greenland ice sheets contain more than 99\% of all the freshwater that is on Earth (National Geographic Society). Due to the fact that ice sheets contain massive quantities of frozen water, if they were to melt, scientists have predicted a sea level rise of about 6 meters for Greenland Ice Sheet and 60 meters for Antarctic Ice Sheets (Ross). The amount of sea level rise due to melting ice sheets depends on the change in its surface mass balance. The surface mass balance is the difference between snow accumulation and snow melting/sublimation (Church et al. 1145).\\

Finally, the third main factor that contributes to rising sea levels is melting glaciers. Glaciers are large masses of ice found near the mountains of most continents and they retreat when their mass balance is negative, similar to ice sheets. Mass balance is the difference in the accumulation of ice in winter versus melting in summer. Since the climate is warming and there is less snowfall, a lot of ice melts in the summer compared to how much accumulates in the winter. This mass loss directly contributes to the rise in sea levels (Church et al. 1145).\\

We were motivated to focus on this research question because it is one of the many visible changes resulting from global warming and climate change. We decided to focus on the visible effects of climate change because we believe that if we can portray its detrimental effects, we will convince a larger audience of its severity. Furthermore, we predict that the reason many humans are not as concerned with climate change is because they do not see its impacts. We hope that if we illustrate the impacts of rising sea levels comparing it against an object that is familiar to everyone, it will resonate with audiences more, leading to more action!


\section*{Dataset Description}

The dataset we are using comes from the Laboratory for Satellite Altimetry/Sea Level Rise. It can be accessed and downloaded from https://www.star.nesdis.noaa.gov/socd/lsa/SeaLevelRise/slr/slr\_sla\_gbl\_keep\_txj1j2\_90.csv. This data is in .csv format. The header is 6 rows long, and there are two rows for the end of the 1992 data. So, we skip over those 8 rows. There are five columns with important information. The first is just the data of the observation. The other four are values for the sea-level rise, with each being from a different altimeter. For each observation, we will be using the value from the latest altimeter (the right-most value that is available).

\section*{Computational Plan}

Our plan for this project is to use the data given about global mean sea-level rise and its trends to predict the future global mean sea-level rise. Firstly, we will find the average change in sea-level rise for each year from the dataset. This will cover the dates from 1993 to 2020. According to NASA's database about global mean sea-levels, the rate of change is 3.3mm per year. We use that to predict sea-level rise from 2020 to 2080. From 2080 to 2100, the global mean sea-level will rise at a rate of approximately 12mm per year (Church et al. 1140); so we will use that rate for 2080 to 2100. The predictions for these trends are accurate when we compare it to research done that shows the sea-level rise until 2100. By 2100, the sea level will rise roughly near the 35cm to 55cm mark (Rahmstorf). Furthermore, since we are now also focusing on the three main factors that contribute to this rise, we will estimate how much each factor contributed to the sea-level rise for each year. After performing calculations on Table 13.1, we find that on average, roughly 41\% of the global mean sea level rise is a result of thermal expansion due to ocean heat contents, 35\% is a result of melting glaciers, and 24\% is a result of melting ice sheets (Church et al. 1151). So, for the sea-level rise we predict for each year, we will apply these percentages to figure out how much each factor contributed to it.\\

We chose to visualize our results using Pygame. To begin, we simply imported the library “pygame” since that is what our whole program is based around. We first had to initialize pycharm using the pygame.init() method so we can actually start using pycharm. We used the pygame.display.set\_mode() method to initialize a window for us to display things on. We used the pygame.image.load() method to upload corresponding images onto our program. We used the pygame.transform.scale() method to literally scale the corresponding image to our desired size. We used the pygame.set\_caption() method to set the title of the window to “Sea Level Rise Simulator”. We used the pygame.font.SysFont() method to set the font of our texts. We used a class named “Button'' to represent a clickable button on our pygame window. It contained functions such as draw, representing a method to literally draw the button and over\_button, representing whether the position of the mouse is over the button or not. 
The pygame.key.get\_pressed() method was used to configure the left and right arrow keys as an interaction feature in our program. The pygame.display.flip() method allowed us to update the display to the user’s screen after changes the user made. An object called display\_surface represented by the pygame.display.set\_mode() method was implemented to create our display. We performed display\_surface.fill() to literally fill the screen with the corresponding colour to display our background. The font.render() was used to draw the corresponding text like year or title, onto the display. The display\_surface.blit() method was used to display the corresponding object, whether it was a picture or text, onto the corresponding position on the display. We also imported “sys” to allow the program to exit from the window itself. We had to add sys.exit() to each section of the program that had code regarding quitting the program. Thirdly, we imported the “time” library that allowed us to keep track of time and update the refresh rate of the display. The pygame.time.Clock() method was used to keep track of time and the clock.tick() method helped to set the desired refresh rate. This concludes the libraries, functions, classes, and method we used and how they helped us complete our desired program. \\

The rise in global mean sea-level is compared to the height of the average male, which is 1.8 metres. We display and scale the male body, and after each year, we will see that the sea-level is covering more and more of the body. Since our data covers change in sea-levels year-to-year, we will be comparing each change to a base height of 0m (the sea-level does not cover any part of the male body yet). Initially, we were going to compare the sea-level rise to the CN Tower but realized that the rise will not be as significant by 2100 in comparison to the height of the CN Tower. We did this so that we can tangibly portray the effects of the rising global sea-level. The program uses the arrow keys to see the changes in the previous or next year. Initially, we had thought of using a slider but learned that Pygame can sense key-presses, so this would be more convenient.\\
Additionally, we have also included the options to view the sea level rise against different objects than the average male body. You can view them in cities that will be greatly impacted by rising sea levels, such as Venice and New York (near the Statue of Liberty). The underlying idea is the same as for the male body, only the comparison has changed.

\newpage
\section*{Instructions for TA}
Firstly, you will need to download the dataset from:\\ https://www.star.nesdis.noaa.gov/socd/lsa/SeaLevelRise/slr/slr\_sla\_gbl\_keep\_txj1j2\_90.csv\\
Name this dataset file 'global\_mean\_sea\_level.csv' and store it inside a folder called 'Datsets'. This folder should be at the same level as the rest of the files downloaded from MarkUs.\\

Then, continue by running the main.py. Once the file is running, a pygame window will show up, representing our home screen titled, “Sea Level Rise Simulator”. The home screen contains 5 options for you to choose from. Each option represents a button that once clicked, you will be taken to the corresponding simulation. Now the “human demo” button represents the comparison between the average height of a male and a female, to the increase in sea level from 1993-2100.
The other four buttons represent simulations for their corresponding cities. Once you are in the corresponding simulation, for example, “Venice Simulation”, you can use the left and right arrow keys on your keyboard to decrease and increase the year respectively. You will begin at the year 1993, and once you click/hold the right arrow key, the year will go up until 2100. You can see which year you are on in the top right corner of the window. As the year goes up, so will the sea level, based on our computations. Once you have reached the year 2100 or wish to leave the corresponding simulation at any time, you can click the back button in the top left corner of the window. This will return you back to the home screen for you to choose from a simulation of your choice. The same interaction can be done with all of the simulations. Once you have checked out every simulation we have to offer, you can exit the window directly or stop the code in the python console. In the console, you will also see that there is a dictionary that maps the years to a list containing change in the global mean sea level change by each factor.

\newpage
\section*{Changes to the Initial Proposal}

Quite a bit of our project has changed since the initial proposal. One of the biggest has to do with the scope and size of the project. Initially, we planned on visualizing the effects of sea-level and diminishing forests. However, our TA suggested focusing on one of the two problems to ensure we will be able to have a complete project in time. Therefore, we decided to stick with the effects of sea-levels, with one minor adjustment. Since sea-levels and sea-level rise varies for different regions on Earth, we decided to focus on the global mean sea-level rise. So, our results may vary if we were to compare it to specific areas of the planet, but we are looking at it from a global scale.\\

One of the most significant changes we made to our proposal is how we predicted the future global mean sea-level rise. Previously, we planned to use historical sea-level trends to predict future sea-level trends. However, our TA noted that this approach may not be very accurate; we cannot strictly use previous data to predict future data. In fact, the calculations and models used to project future sea-level rise are very complex. So, we are focusing on the three main contributors to global sea-level rise (ocean heat capacity, melting ice sheets, and melting glaciers). There is already data available about the future global mean sea level rise, so now we will research how much each factor roughly contributes to the global sea-level rise.

\section*{Discussion}
By our computations, we figured out the rising sea level data until 2100, and how much its 3 main factors, ocean heat capacity, melting glaciers and melting ice sheets, contributed towards the increase. Based on our research, we concluded that the rising sea levels will contribute towards floods, erosion, coral bleaching, etc, due to its main factors. Our computations confirmed that, in fact, the sea level will continue to rise until 2100 and possibly beyond that point. By figuring out the impact of the three main factors of rising sea levels, in our computations, our research was made even more concrete and allowed us to conclude that in fact Earth will be altered drastically by the rising sea levels. Overall, we believe our research represented a claim that was eventually backed up by our computations on predicting the future increase of sea levels and figuring out what impact these factors may have had on this increase.\\

One very general limitation to the data was that the data found did not provide predictions beyond 2020. Although that would defeat the computational purpose of the assignment, since scientists would spend months, even years researching topics such as thermal expansion, their future predictions would end up being more accurate compared to our estimations. This would allow our sea level data to contain a smaller error margin. The pygame library was not enough to allow us to exit the window directly, so we imported a library called “sys”. We used the sys.exit() method in sections of the program containing the “quit” code to counteract this limitation. We had difficulties regarding resizing the window algorithm since we could not simply copy someone’s code from the internet. The pygame website did not provide enough insight as to how to scale images and objects inside of the window but it provided information on how to resize the window alone. This was not enough information for us to go on, so we decided to focus on displaying our computations instead of spending hours on the resizing algorithm. This was our main obstacle we encountered during our programming phase. Finally, another major obstacle was finding certain images of cities that were in risk of rising sea level, that could be scaled based on our sea level predictions.\\

Now that we have analyzed the increase of sea levels and it’s factors, to further our exploration, the next step should be to research climate change itself. We know rising sea levels is an effect of climate change. We focused our exploration on an aspect of climate change but now it would be time to tackle the topic itself. We based our causes of rising sea levels on three major factors. Similarly, for climate change as a whole, we would consider its own effects on Earth. Specifically, factors such as global temperature increase, change in precipitation patterns, a major increase in droughts and heat waves, rising sea levels, stronger hurricanes and many more. Diving deeper, we would have to analyze each factor’s effects themselves to analyze their general trend and how devastating each factor may be. This exploration on climate change as a whole would take an extensive amount of research as it has so many factors that come into play, each of which containing sub-factors of their own. Now, the general project would be based around the Earth’s surface, in the future, after climate change itself instead of just rising sea levels. The exploration would try to answer a question like: "What would happen to Earth if climate change is not reduced or stopped?"
\newpage
\section*{References}

“About - Wiki.” About - Pygame Wiki, www.pygame.org/wiki/about.\\

Church, J.A., P.U. Clark, A. Cazenave, J.M. Gregory, S. Jevrejeva, A. Levermann, M.A. Merrifield, G.A. Milne, R.S. Nerem, P.D. Nunn, A.J. Payne, W.T. Pfeffer, D. Stammer and A.S. Unnikrishnan, 2013: Sea Level Change. In: Climate Change 2013: The Physical Science Basis. Contribution of Working Group I to the Fifth Assessment Report of the Intergovernmental Panel on Climate Change [Stocker, T.F., D. Qin, G.-K. Plattner, M. Tignor, S.K. Allen, J. Boschung, A. Nauels, Y. Xia, V. Bex and P.M. Midgley (eds.)]. Cambridge University Press, Cambridge, United Kingdom and New York, NY, USA.\\

Dahlman, LuAnn, and Rebecca Lindsey. “Climate Change: Ocean Heat Content: NOAA Climate.gov.” Climate Change: Ocean Heat Content | NOAA Climate.gov, 17 Aug. 2020, www.climate.gov/news-features/understanding-climate/climate-change-ocean-heat-content.\\

National Geographic Society. “Ice Sheet.” National Geographic Society, 9 Oct. 2012,\\ www.nationalgeographic.org/encyclopedia/ice-sheet/.\\

Rahmstorf, Stefan. “Modeling Sea Level Rise.” Nature News, Nature Publishing Group, 2012,\\ www.nature.com/scitable/knowledge/library/modeling-sea-level-rise-25857988/.\\

Ross, Rachel. “What Are the Different Types of Ice Formations Found on Earth?” LiveScience, Purch, 8 Jan. 2019, www.livescience.com/64444-ice-formations.html.\\


\end{document}
